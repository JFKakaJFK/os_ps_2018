% !TEX TS-program = pdflatex
% !TEX encoding = UTF-8 Unicode
\documentclass[11pt]{article} % use larger type; default would be 10pt
\usepackage[utf8]{inputenc} % set input encoding (not needed with XeLaTeX)
\usepackage{geometry} % to change the page dimensions
\geometry{a4paper} % or letterpaper (US) or a5paper or....
\usepackage{verbatim} % adds environment for commenting out blocks of text & for better verbatim
\usepackage{hyperref}
\usepackage{listings}
\usepackage{color}

\lstdefinestyle{custom}{
  basicstyle=\footnotesize\ttfamily,
  backgroundcolor=\color{white},
  keywordstyle=\color{magenta},
  commentstyle=\color{black},
  stringstyle=\color{cyan},
  numbers=left,
  numbersep=.5em,
  breakatwhitespace=false,
  breaklines=true,
  showstringspaces=false,
  keepspaces=true,
  showspaces=false,
  showtabs=false,
  tabsize=2
}
\lstset{style=custom}

\title{Excercise Sheet 2}
\author{Johannes Koch}
%\date{} % Activate to display a given date or no date (if empty),
         % otherwise the current date is printed

\begin{document}
\maketitle

\section{Task 1}

\subsection{Script 1}

\lstinputlisting[language=bash]{script1.sh}
Line 1: tell the shell to use \texttt{/bin/bash} as interpreter\\
Line 2-3: comments\\
Line 4-5, 7: for loop, executes Line 6 for every given commandline argument\\
Line 6: change the access permissions for the current argument to \texttt{-rwxr-x--- ...}\\
\\
Execution with filenames or directories as commandline arguments:\\
The script will change the access permissions for each valid argument or throw an error to stdout.\\
\\
Execution without any arguments:\\
Since the for-loop will never execute the script will do nothing.\\
\pagebreak
\subsection{Script 2}

\lstinputlisting[language=bash]{script2.sh}
Line 1: tell the shell to use \texttt{/bin/bash} as interpreter\\
Line 2: comment\\
Line 3-4: if the number of commandline arguments is less than 3, execute Lines 5-7\\
Line 5: print the string \texttt{"Error. ..."} to \texttt{stderr} while
interpreting escape characters\\
Line 6: same as Line 5 with other text
Line 7: exit with \href{http://www.tldp.org/LDP/abs/html/exitcodes.html}{error code 1
 (Catchall for general errors)}\\
Line 8-9: if the number of commandline arguments is greater than 3, execute Lines 10-12\\
Line 10-11: same as Line 5 with other error message\\\
Line 12: exit with \href{http://www.tldp.org/LDP/abs/html/exitcodes.html}{error
 code 2 (Misuse of shell builtins(according to Bash documentation))}\\
Line 13: if none of the above cases occur, execute Line 14\\
Line 14: prints "Argument count correct. Proceeding..." and a newline to \texttt{stdout}\\
Line 15: end the if clause\\
\\
Execution with less than 3 commandline arguments:\\
Lines 5-6 will be printed to \texttt{stderr} and the program will exit with error code 1\\
\\
Execution with more than 3 commandline arguments:\\
Lines 10-11 will be printed to \texttt{stderr} and the program will exit with error code 2\\
\\
Execution with 3 commandline arguments:\\
"Argument count correct. Proceeding..." and a newline command will be printed to \texttt{stdout}\\
\pagebreak
\subsection{Script 3}

\lstinputlisting[language=bash]{script3.sh}
Line 1: tell the shell to use \texttt{/bin/bash} as interpreter\\
Line 2: comment\\
Line 3: \texttt{INFILE} is now the first commandline argument\\
Line 4: \texttt{OUTFILE} is now the second commandline argument\\
Line 5-6: if the file \texttt{INFILE} exists execute Lines 7-12\\
Line 7-8: if the \texttt{OUTFILE} is a writeable file, execute Line 9\\
Line 9: append the content of \texttt{INFILE} to \texttt{OUTFILE}\\
Line 10: if \texttt{OUTFILE} is not a writeable file, execute Line 11\\
Line 11: print "can not write to " and the second argument to \texttt{stdout}\\
Line 12: end the inner if clause\\
Line 13: if \texttt{INFILE} does not exist, execute Line 14\\
Line 14: prints "can not read from" and the first argument to \texttt{stdout}\\
Line 15: end the outer if clause\\
\\
Execution with at least 2 arguments, where the first one is an existing file
 and the second a writeable file:\\
The content of the first file will be appended to the second file\\
\\
Execution with at least 1 argument, where the first one is an existing file:\\
"can not write to " and the second argument or "" will be printed to \texttt{stdout}\\
\\
Execution with any number of arguments, where the first one doesn't exist:\\
"can not read from " and the first argument or "" will be printed to \texttt{stdout}\\
\pagebreak

\section{Task 2}
\lstinputlisting[language=bash]{my_back_up.sh}
\end{document}
