% !TEX TS-program = pdflatex
% !TEX encoding = UTF-8 Unicode
\documentclass[11pt]{article} % use larger type; default would be 10pt
\usepackage[utf8]{inputenc} % set input encoding (not needed with XeLaTeX)
\usepackage{geometry} % to change the page dimensions
\geometry{a4paper} % or letterpaper (US) or a5paper or....
\usepackage{verbatim} % adds environment for commenting out blocks of text & for better verbatim
\usepackage[dvipsnames]{xcolor}
\usepackage{listings}

\lstdefinestyle{custom}{
  basicstyle=\footnotesize\ttfamily,
  backgroundcolor=\color{White},
  keywordstyle=\color{Cerulean},
  commentstyle=\color{Green},
  stringstyle=\color{Purple},
  %numbers=left,
  %numbersep=.5em,
  breakatwhitespace=false,
  breaklines=true,
  showstringspaces=false,
  keepspaces=true,
  showspaces=false,
  showtabs=false,
  tabsize=2
}
\lstset{style=custom}

\title{Excercise Sheet 7}
\author{Johannes Koch}
%\date{} % Activate to display a given date or no date (if empty),
         % otherwise the current date is printed

\begin{document}
\maketitle

\section{Excercise Sheet 6 Task 2}
\lstinputlisting[language=C]{e6task2.c}

\section{Task 1}
In theory when a mutex unsuccessfully tries to lock a mutex it will go to sleep, allowing other processes to run
immediately and it will retry when it is woken by another thread which unlocked the mutex.\\
When a thread unsuccessfully tries to lock a spinlock it will continue to retry until it successfully locks the spinlock
or the threads CPU runtime quantum has been exceeded. This results in busy waiting.\\
If the critical region is only locked for a short time, putting a thread to sleep and waking it again creates a lot
of overhead. A spinlock on the other hand would create less overhead but would use 100\% of the CPU for a short amount of time.
If the critical region is locked for a considerable amount of time or if the System only has a single CPU-Core spinlocks will needlessly
occupy recources another thread could have used if mutexes were used.

\lstinputlisting[language=C]{task1.c}

\subsection{/usr/bin/time}


\section{Task 1}
Using condition variables and mutexes naturally creates more overhead than simply using mutexes or spinlocks, but
on the upside a lot more control is gained, for example in a reader-writer/consumer-producer scenarion it can be controlled
which side gets access to the critical region.

\lstinputlisting[language=C]{task2.c}

\subsection{/usr/bin/time}

\end{document}
