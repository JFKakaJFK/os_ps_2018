% !TEX TS-program = pdflatex
% !TEX encoding = UTF-8 Unicode
\documentclass[11pt]{article} % use larger type; default would be 10pt
\usepackage[utf8]{inputenc} % set input encoding (not needed with XeLaTeX)
\usepackage{geometry} % to change the page dimensions
\geometry{a4paper} % or letterpaper (US) or a5paper or....
\usepackage{verbatim} % adds environment for commenting out blocks of text & for better verbatim

\usepackage{listings}
\usepackage[dvipsnames]{xcolor}

\lstdefinestyle{custom}{
  basicstyle=\footnotesize\ttfamily,
  backgroundcolor=\color{White},
  keywordstyle=\color{Cerulean},
  commentstyle=\color{Gray},
  stringstyle=\color{OliveGreen},
  %numbers=left,
  %numbersep=.5em,
  breakatwhitespace=false,
  breaklines=true,
  showstringspaces=false,
  keepspaces=true,
  showspaces=false,
  showtabs=false,
  tabsize=2
}
\lstset{style=custom}

\title{Excercise Sheet 4}
\author{Johannes Koch}
%\date{} % Activate to display a given date or no date (if empty),
         % otherwise the current date is printed

\begin{document}
\maketitle

\section{Task 1}

Compile \texttt{reader.c} and \texttt{task1.c}:\\
\texttt{gcc -Wall -Werror -std=c99 reader.c -o reader}\\
\texttt{gcc -Wall -Werror -std=c99 task1.c -o task1}\\
and call:\\
\texttt{./reader \& ./task1} or each of the binarys in a separate console window.

\subsection{\texttt{reader.c}}
\lstinputlisting[language=C]{reader.c}

\subsection{\texttt{task1.c}}
\lstinputlisting[language=C]{task1.c}

\clearpage

\section{Task 2}
Task 2 uses the reader of task 1, \texttt{task2a.c} and \texttt{task2b.c}
both need the additional \texttt{-pthread} compile flag. To execute the program,
call \texttt{./reader \& ./task2a}/\texttt{./reader \& ./task2b} or each of the binarys
in a separate console window.\\
The semaphores in \texttt{task2a} seem to be very unreliable(no idea what was done wrong),
thats why \texttt{task2b} exists.

\subsection{\texttt{task2a.c}}
\lstinputlisting[language=C]{task2a.c}

\subsection{\texttt{task2b.c}}
\lstinputlisting[language=C]{task2b.c}

\end{document}
